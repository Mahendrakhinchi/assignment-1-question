\documentclass[a4paper,12pt]{article}
\usepackage{graphicx}
\usepackage{circuitikz}
\usepackage{amsmath}

\begin{document}

\title{Experimental Verification of KVL and KCL}
\author{Mahendra Khinchi}
\date{\today}
\maketitle

\section{Aim}
The aim of this experiment is to experimentally verify Kirchhoff's Voltage Law (KVL) and Kirchhoff's Current Law (KCL) using resistances and a voltage source.

\section{Apparatus Required}
\begin{itemize}
    \item Breadboard
    \item Resistors (1k to 100k ohm range)
    \item Voltage source
    \item Connecting wires
    \item Multimeter
\end{itemize}
\section{Circuit Diagram}
\begin{figure}[h]
    \centering
    \begin{circuitikz}
        \draw (0,0) to [V, v=$V_s$] (0,3)
        to [R, l=$R_1$] (3,3)
        to [R, l=$R_2$] (3,0)
        -- (0,0);
    \end{circuitikz}
    \caption{Experimental circuit diagram}
    \label{fig:circuit_diagram}
\end{figure}

\section{Theory}
Kirchhoff's Voltage Law (KVL) states that the algebraic sum of the potential differences in any closed loop in a circuit is zero. Mathematically, it can be expressed as:
\[
V_s = V_{R_1} + V_{R_2}
\]

Kirchhoff's Current Law (KCL) states that the sum of currents entering a node is equal to the sum of currents leaving the node. Mathematically, for the node between $R_1$ and $R_2$:
\[
I_{\text{in}} = I_{R_1} + I_{R_2}
\]

\section{Calculation}
Assuming a voltage $V_s$, calculate the currents and voltages across $R_1$ and $R_2$ using Ohm's Law. Verify KVL and KCL for the given circuit.

\section{Observation}
Record the measured values of voltages and currents using the multimeter. Compare the observed results with the calculated values.

\section{Conclusion}
Summarize the results obtained from the experiment. Discuss any discrepancies and possible sources of error.

\end{document}

