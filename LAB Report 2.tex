umentclass{article}
\usepackage{pgfplots}

\begin{document}

\begin{figure}
  \centering
  \begin{tikzpicture}
    \begin{axis}[
      xlabel={Frequency (Hz)},
      ylabel={Gain (dB)},
      title={Low-Pass RC Filter Frequency Response},
      grid=major,
      legend entries={Simulation, Theory},
      legend pos=north east,
    ]
    
    % Randomly select values for R and C
    \pgfmathsetmacro{\Rvalue}{random(1000, 1000000)}
    \pgfmathsetmacro{\Cvalue}{random(100e-12, 100e-9)}
    
    % Calculate cutoff frequency
    \pgfmathsetmacro{\cutoffFreq}{1 / (2 * pi * \Rvalue * \Cvalue)}
    
    % Plot simulation data
    \addplot[blue, domain=1:1000000, samples=100, thick] {20 * log10(1 / sqrt(1 + (x * \Cvalue * \Rvalue)^2))};
    
    % Plot theoretical cutoff frequency
    \draw[dashed, red] ({\cutoffFreq}, -3) -- ({\cutoffFreq}, 3);
    \node[red, below] at ({\cutoffFreq}, -3) {\scriptsize{$f_{\text{cutoff}}$}};
    
    \end{axis}
  \end{tikzpicture}
  \caption{Random Low-Pass RC Filter Frequency Response}
\end{figure}

\end{document}